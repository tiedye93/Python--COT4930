
\chapter{re -- Regular Expressions}

\section{re -- Regular expression operations}

This module provides regular expression
matching operations similar to those found in Perl.

Both patterns and strings to be searched
can be Unicode strings as well as 8-bit
strings. However, Unicode strings and
8-bit strings cannot be mixed: that is,
you cannot match an Unicode string with
a byte pattern or vice-versa; similarly,
when asking for a substitution, the replacement string must
be of the same type as both the pattern and the search string.


\subsection{Regular Expression Syntax}

A regular expression (or RE) specifies
a set of strings that matches it; the
functions in this module let you check
if a particular string matches a given
regular expression (or if a given regular
expression matches a particular string,
which comes down to the same thing).

Some characters, like \verb+'|'+ or \verb+'('+, are
special. Special characters either stand
for classes of ordinary characters, or
affect how the regular expressions around
them are interpreted. Regular expression
pattern strings may not contain null bytes,
but can specify the null byte using a \verb+\number+
notation such as \verb+'\x00'+.

The special characters are:

\begin{description}

\item[\texttt{'.'}]
(Dot.) In the default mode, this matches
any character except a newline. If the
DOTALL flag has been specified, this matches
any character including a newline.

\item[\texttt{'\$'}]
Matches the end of the string or just
before the newline at the end of the string,
and in MULTILINE mode also matches before
a newline. \verb+foo+ matches both \verb+'foo'+ and
\verb+'foobar'+, while the regular expression
\verb+foo$+ matches only \verb+foo'+. More interestingly,
searching for \verb+foo.$+ in \verb+'foo1\nfoo2\n'+
matches \verb+'foo2'+ normally, but \verb+'foo1'+ in
MULTILINE mode; searching for a single
\verb+$+ in \verb+'foo\n'+ will find two (empty) matches:
one just before the newline, and one at the end of the string.

\pagebreak
\item[\texttt{[ ]}]
Used to indicate a set of characters. In a set:

\begin{itemize}
\item Characters can be listed individually,
e.g. [amk] will match \verb+'a'+, \verb+'m'+, or \verb+'k'+.

\item Ranges of characters can be indicated
by giving two characters and separating
them by a \verb+'-'+, for example \verb+[a-z]+ will
match any lowercase ASCII letter, \verb+[0-5][0-9]+
will match all the two-digits numbers
from 00 to 59, and [0-9A-Fa-f] will match
any hexadecimal digit. If - is escaped
(e.g. \verb+[a\-z]+) or if it is placed as the
first or last character (e.g. \verb+[a-]+),
it will match a literal \verb+'-'+.
\end{itemize}

\end{description}

